\documentclass[bigtut]{tutorial}\usepackage[]{graphicx}\usepackage[]{color}
%% maxwidth is the original width if it is less than linewidth
%% otherwise use linewidth (to make sure the graphics do not exceed the margin)
\makeatletter
\def\maxwidth{ %
  \ifdim\Gin@nat@width>\linewidth
    \linewidth
  \else
    \Gin@nat@width
  \fi
}
\makeatother

\definecolor{fgcolor}{rgb}{0.345, 0.345, 0.345}
\newcommand{\hlnum}[1]{\textcolor[rgb]{0.686,0.059,0.569}{#1}}%
\newcommand{\hlstr}[1]{\textcolor[rgb]{0.192,0.494,0.8}{#1}}%
\newcommand{\hlcom}[1]{\textcolor[rgb]{0.678,0.584,0.686}{\textit{#1}}}%
\newcommand{\hlopt}[1]{\textcolor[rgb]{0,0,0}{#1}}%
\newcommand{\hlstd}[1]{\textcolor[rgb]{0.345,0.345,0.345}{#1}}%
\newcommand{\hlkwa}[1]{\textcolor[rgb]{0.161,0.373,0.58}{\textbf{#1}}}%
\newcommand{\hlkwb}[1]{\textcolor[rgb]{0.69,0.353,0.396}{#1}}%
\newcommand{\hlkwc}[1]{\textcolor[rgb]{0.333,0.667,0.333}{#1}}%
\newcommand{\hlkwd}[1]{\textcolor[rgb]{0.737,0.353,0.396}{\textbf{#1}}}%
\let\hlipl\hlkwb

\usepackage{framed}
\makeatletter
\newenvironment{kframe}{%
 \def\at@end@of@kframe{}%
 \ifinner\ifhmode%
  \def\at@end@of@kframe{\end{minipage}}%
  \begin{minipage}{\columnwidth}%
 \fi\fi%
 \def\FrameCommand##1{\hskip\@totalleftmargin \hskip-\fboxsep
 \colorbox{shadecolor}{##1}\hskip-\fboxsep
     % There is no \\@totalrightmargin, so:
     \hskip-\linewidth \hskip-\@totalleftmargin \hskip\columnwidth}%
 \MakeFramed {\advance\hsize-\width
   \@totalleftmargin\z@ \linewidth\hsize
   \@setminipage}}%
 {\par\unskip\endMakeFramed%
 \at@end@of@kframe}
\makeatother

\definecolor{shadecolor}{rgb}{.97, .97, .97}
\definecolor{messagecolor}{rgb}{0, 0, 0}
\definecolor{warningcolor}{rgb}{1, 0, 1}
\definecolor{errorcolor}{rgb}{1, 0, 0}
\newenvironment{knitrout}{}{} % an empty environment to be redefined in TeX

\usepackage{alltt}
\unitcode{MATH1005}
        \unitname{Statistics}
        \semester{Summer/Winter/Semester2}
        \sheetnumber10

\usepackage{graphicx}
\withsolutions
\IfFileExists{upquote.sty}{\usepackage{upquote}}{}
\begin{document}
\lettersfirst

\begin{tutorial}

\begin{center}
\begin{tabular}{| ll |} \hline
& \\
{\bf Proportion Test} & \\
Context & $n$ independent Binary trials with unknown success probability $\rho$ \\
Hypothesis &   $H_{0}: \rho = \rho_{0}$   \\
Test Statistic & $X=$ \# successes $  \overset{H_0}{\sim} Bin(n, \rho_{0})$      \\ 
& \\  
& Note: For large $n$, $X$ can be approximated by a Normal, with continuity correction. \\ 
& \\ \hline
\end{tabular}
\end{center}


\begin{center}
\begin{tabular}{| ll |} \hline
& \\
{\bf Sign Test} & \\
Context &  $n$ observations $ \{ x_{i} \} $ from a continuous distribution with unknown mean $\mu$ \\
& We want to test $H_{0}: \mu = \mu_{0}$ \\
Working & Work out signs $ \{ x_{i} -\mu_{0} \}$ (eliminating any zeroes)  \\
Hypothesis &   $H_{0}: \rho_{+} =0.5$   \\
Test Statistic & $X=$ \# + signs$  \overset{H_0}{\sim} Bin(n, 0.5)$      \\ 
& \\  \hline
\end{tabular}
\end{center}



 \begin{questions}

\vspace{.5cm}
\question  Hypothesis Testing  \\

In your own words, explain the following concepts: \\

\begin {parts}
\part The purpose of hypothesis testing 
\part The importance of assumptions
\part Test statistic
\part P-value
\end{parts}


\begin{solution}
(a)The purpose of hypothesis testing is to make a robust decision about an unknown population parameter (specified in $H_{0}$ vs $H_{1}$), based on an observed sample.  It allows evidence based decision making. \\

(b)  A hypothesis test is invalid if the assumptions are not satisfied. We weigh up the validity of assumptions to determine what test to choose, given a choice. \\

(c) A test statistic is a random variable consisting of a function of the observed values, with a distribution depending on the unknown parameter. \\

(c) The p-value is the weighting of the evidence for $H_{1}$ from the sample, assuming $H_{0}$ is true.
\end{solution}


\question  Orange juice consumption  \\

For the last decade in the USA, orange juice sales have fallen almost each year, with per-capita consumption falling roughly 40\%. \\
{\tiny http://qz.com/176096/how-america-fell-out-of-love-with-orange-juice/} \\

Consider a certain city, where the proportion of families buying a certain brand of orange is believed to be 0.3. A random sample of 50 families from this city shows that only 11 bought that brand of juice.  \\

Let $X$ be the number of families in the sample buying the brand and let $\rho$ be the probability that a randomly picked family prefers the brand. \\

\begin{parts}
\part Hypothesis: Explain why a one-sided test of $H_0: \rho=0.3  \;\; vs \;\; H_1: \rho<0.3$
 is appropriate here. \\
 
\part Assumptions: To model $X$ as  a Binomial random variable, what further assumptions are needed?\\

 \part Test statistic: What is the test statistic and its distribution under $H_{0}$? \\
 
\part P-value: Calculate the p-value using R.

\begin{knitrout}
\definecolor{shadecolor}{rgb}{0.969, 0.969, 0.969}\color{fgcolor}\begin{kframe}
\begin{alltt}
\hlkwd{pbinom}\hlstd{(}\hlnum{11}\hlstd{,}\hlnum{50}\hlstd{,}\hlnum{0.3}\hlstd{)}
\end{alltt}
\end{kframe}
\end{knitrout}

\part Conclusion: Draw your conclusion based on the p-value.
\end{parts}


\begin{solution}
(a) 
\fbox{H} $H_0: \rho=0.3  \;\; vs \;\; H_1: \rho<0.3$ \\
We choose a one-sided alternative because the information goven about sales in the USA suggests that the brand is now {\it less} popular than before. \\
 
(b)
\fbox{A} We assume each of the 50 families are sampled independently and have the same preference $\rho$ for the orange juice brand. \\

(c)
\fbox{T} $X = \text{Number of families buying the brand} \overset{H_0}{\sim} Bin(50, 0.3)$. 

The observed value is $x=11$. \\
 
(d) 
\fbox{P}  p-value = $P(X \leq 11) = 0.1390361$. \\

(e)
\fbox{C} As p-value is about 14\%, the evidence supports $H_{0}$. \\
 Hence it appears that this brand has not lost its market share.

\end{solution}






\question Moisture retention \\

The following data are 8 measurements of moisture
retention (\%) using a new scaling system. This system is expected
to be better (ie have greater retention) than the system previously
in use, for which the mean retention was 96\%. 

\begin{knitrout}
\definecolor{shadecolor}{rgb}{0.969, 0.969, 0.969}\color{fgcolor}\begin{kframe}
\begin{alltt}
\hlstd{x}\hlkwb{=}\hlkwd{c}\hlstd{(}\hlnum{97.5}\hlstd{,}\hlnum{96.2}\hlstd{,}\hlnum{97.3}\hlstd{,}\hlnum{96.0}\hlstd{,}\hlnum{99.8}\hlstd{,}\hlnum{93.0}\hlstd{,}\hlnum{94.2}\hlstd{,}\hlnum{95.5}\hlstd{)}
\end{alltt}
\end{kframe}
\end{knitrout}

\vspace{.5cm}
\begin{parts}
\part Preparation: By comparing the data to 96, write down the signs.\\

\part Hypothesis: Explain why $H_0: \rho=0.5  \;\; vs \;\; H_1: \rho>0.5$
is the appropriate hypothesis. \\

\part Assumptions: To perform a sign test, what assumptions are needed? \\

\part Test statistic: What is the test statistic and its distribution under $H_{0}$? \\

\part P-value: Calculate the p-value using Binomial Tables.  \\

\part Conclusion: Draw your conclusion based on the p-value.
\end{parts}


\begin{solution}
(a)
\fbox{Preparation}
$\{ \text{Signs} \} = \{ + + +  \; 0 + - - - \}$. 
(We discard the 0).  \\

(b)
\fbox{H} 
$H_0: \mu=0.96  \;\; vs \;\; H_1: \mu>0.96$
is equivalent to  
$H_0: \rho=0.5  \;\; vs \;\; H_1: \rho>0.5$ \\
where $\rho = P(\mbox{Measurement} > 0.96)$.\\
We use a one-sided test as the system is expected to be better. \\

(c)
\fbox{A} We assume the data is from a continuous distribution. \\

(d)
\fbox{T} 
$X = \text{Number of $+$ signs} \overset{H_0}{\sim} Bin(7,0.5)$.\\
Observed value: $x=4$. \\

(e)
\fbox{P}
$\text{P-value} = P(X \geq 4) = 1- P(X \leq 3) 
= 1- 0.5 = 0.5$. 

\begin{knitrout}
\definecolor{shadecolor}{rgb}{0.969, 0.969, 0.969}\color{fgcolor}\begin{kframe}
\begin{alltt}
\hlnum{1}\hlopt{-}\hlkwd{pbinom}\hlstd{(}\hlnum{3}\hlstd{,}\hlnum{7}\hlstd{,}\hlnum{0.5}\hlstd{)}
\end{alltt}
\begin{verbatim}
## [1] 0.5
\end{verbatim}
\end{kframe}
\end{knitrout}

(f)
\fbox{C} Given the large $p$=value, we would expect this data 50\% of the time, and hence the evidence is consistent with $H_{0}$. \\
Hence, it appears that the new system has  not improved. \\
Note: This was based on a small sample. Increasing the sample size might change the result.
\end{solution}








\newpage
\hspace{-1cm} {\bf Extra Questions}


\question New technique (paired data) \\

A new measuring technique is being considered to replace the standard technique. When 10 samples are measured by both techniques, the measurements are \\
\begin{center}
\begin{tabular}{|l|cccccccccc|}\hline
 Sample\/: &1&2&3&4&5&6&7&8&9&10\\ \hline
New      &2.5 &2.2 &2.6 &2.6 &1.9 &3.3 &3.3 &2.8 &3.0 &2.9\\
Standard &2.1 &2.4 &2.1 &1.9 &2.0 &2.8 &2.7 &2.8 &2.8 &3.0\\ \hline
\end{tabular}
 \end{center}

\vspace{.5cm}
Test the hypothesis that there is no long-run systematic difference between the two techniques. \\

Hint: First, calculate Difference = New-Standard. Second, write down the signs of the Differences. Third, perform the sign test.

\begin{solution}
\fbox{Preparation} \\

\begin{tabular}{|l|cccccccccc|}\hline
 Sample\/: &1&2&3&4&5&6&7&8&9&10\\ \hline
New      &2.5 &2.2 &2.6 &2.6 &1.9 &3.3 &3.3 &2.8 &3.0 &2.9\\
Standard &2.1 &2.4 &2.1 &1.9 &2.0 &2.8 &2.7 &2.8 &2.8 &3.0 \\ 
Differences = New - Standard & 0.4 & -0.2 & 0.5 & 0.7 & -0.1 & 0.5 & 0.6 & 0 & 0.2 & -0.1 \\
Signs & + & - & + & + & - & + & + & 0 & + & - \\ \hline
\end{tabular}

\vspace{.3cm}
So we have a sample of signs of size $n=9$ with $x=\text{number of + signs}= 6$. \\

Define $X = \text{Number of + signs} \sim Bin(9,\rho)$, where $\rho = P(+ sign) = P(\text{New different than Standard})$. \\

\fbox{H} 
$H_0: \rho=0.5  \;\; vs \;\; H_1: \rho \neq 0.5$ \\

\fbox{A} We assume the measurements were sampled independently with same $\rho$. \\

\fbox{T} 
$X \overset{H_0}{\sim} Bin(9,0.5)$.\\
Observed value: $x=6$. \\

\fbox{P}
$\text{P-value} = 2 P(X \geq 6) = 2(1- P(X \leq 5)) 
= 2(1- 0.7461) = 0.5078$. \\

\begin{knitrout}
\definecolor{shadecolor}{rgb}{0.969, 0.969, 0.969}\color{fgcolor}\begin{kframe}
\begin{alltt}
\hlnum{2}\hlopt{*}\hlstd{(}\hlnum{1}\hlopt{-}\hlkwd{pbinom}\hlstd{(}\hlnum{5}\hlstd{,}\hlnum{9}\hlstd{,}\hlnum{0.5}\hlstd{))}
\end{alltt}
\begin{verbatim}
## [1] 0.5078125
\end{verbatim}
\end{kframe}
\end{knitrout}

\fbox{C} As $\text{p-value}$ is large, the evidence is consistent with $H_{0}$. \\
Hence, it appears that the standard and new techniques are not systematically different.
\end{solution}






\question Gender of children \\

A certain family has 7 children and they are all girls.
Perform a 2 sided test of the hypothesis that in that family each child is either a boy or
a girl independently with equal probability. Write all your steps. Calculate the p-value using both Binomial tables and R. 

\begin{solution}
Let $X = \text{the number of girls in family of 7}$. Then $X \sim Bin(7, \rho)$, where $\rho = P(girl)$. \\

\fbox{H} $H_0: \rho=0.5  \;\; vs \;\; H_1: \rho \neq 0.5$ \\
 
\fbox{A} We assume that the gender of each of the children is independent of the others and that they have the  same probability $\rho = P(girl)$.\\

\fbox{T} $X = \text{Number of girls} \overset{H_0}{\sim} Bin(7, 0.5)$. 

The observed value is $x=7$. \\
 
\fbox{P} \\

Using Formula:\\
$\text{p-value} = 2 P(X \geq 7) = 2 P(X=7) = 2 {7 \choose 7} (0.5)^7(0.5)^0  = 0.015625$. \\

Using Binomial Tables:\\
$\text{p-value} = 2 P(X \geq 7) = 2 (1-P(X\leq 6)) =2 (1-0.992) = 0.016$. \\

Using R: \\
$\text{p-value} \doteq 0 $
\begin{knitrout}
\definecolor{shadecolor}{rgb}{0.969, 0.969, 0.969}\color{fgcolor}\begin{kframe}
\begin{alltt}
\hlnum{2}\hlopt{*}\hlstd{(}\hlnum{1}\hlopt{-}\hlkwd{pbinom}\hlstd{(}\hlnum{6}\hlstd{,}\hlnum{7}\hlstd{,}\hlnum{0.5}\hlstd{))}
\end{alltt}
\begin{verbatim}
## [1] 0.015625
\end{verbatim}
\end{kframe}
\end{knitrout}

\vspace{.5cm}
\fbox{C} As p-value $ < < 0.05$, we have evidence against $H_{0}$ at $\alpha =0.05$. \\
Hence there is evidence against the family having girls with $\rho = 0.5$. \\

\end{solution}


\question  Taste-Testing \\

14 students taste-tested two different brands of drink
(brand $A$ and brand $B$), with the brands being hidden from them. The
object of the exercise was to see if students preferred one brand
over the other, but there was no indication of which this might be
before the test. Overall, 8 subjects preferred brand $A$, 4 preferred
brand $B$ and 2 had no preference either way. Use a sign test to test whether there is a difference between the 2 brands.
Write all your steps. Calculate the p-value using Binomial tables.


\begin{solution}
$X = \text{Number of $+$ signs} \sim Bin(12, p)$, where $\rho = P(\text{preference for brand }A)$ and $x=8$. \\
Note: 2 readings were discarded (`no preference'), giving us $n=12$. \\

\fbox{H} 
$H_0: \rho=0.5  \;\; vs \;\; H_1: \rho \neq 0.5$ \\

\fbox{A} We assume the students were sampled independently with same $\rho$. \\

\fbox{T} 
$X \overset{H_0}{\sim} Bin(12,0.5)$.\\
Observed value: $x=8$. \\

\fbox{P}
$\text{P-value} = 2 P(X \geq 8) = 2(1- P(X \leq 7) 
= 2(1- 0.8062) = 0.3876$. \\

\begin{knitrout}
\definecolor{shadecolor}{rgb}{0.969, 0.969, 0.969}\color{fgcolor}\begin{kframe}
\begin{alltt}
\hlnum{2}\hlopt{*}\hlstd{(}\hlnum{1}\hlopt{-}\hlkwd{pbinom}\hlstd{(}\hlnum{7}\hlstd{,}\hlnum{12}\hlstd{,}\hlnum{0.5}\hlstd{))}
\end{alltt}
\begin{verbatim}
## [1] 0.3876953
\end{verbatim}
\end{kframe}
\end{knitrout}

\fbox{C} As $\text{p-value}$ is large, the evidence is consistent with $H_{0}$. \\
Hence, there does not appear to be a differnece between the brands. \\
\end{solution}











\end{questions}
\end{tutorial}
\end{document}
