\documentclass[bigtut]{tutorial}\usepackage[]{graphicx}\usepackage[]{color}
%% maxwidth is the original width if it is less than linewidth
%% otherwise use linewidth (to make sure the graphics do not exceed the margin)
\makeatletter
\def\maxwidth{ %
  \ifdim\Gin@nat@width>\linewidth
    \linewidth
  \else
    \Gin@nat@width
  \fi
}
\makeatother

\definecolor{fgcolor}{rgb}{0.345, 0.345, 0.345}
\newcommand{\hlnum}[1]{\textcolor[rgb]{0.686,0.059,0.569}{#1}}%
\newcommand{\hlstr}[1]{\textcolor[rgb]{0.192,0.494,0.8}{#1}}%
\newcommand{\hlcom}[1]{\textcolor[rgb]{0.678,0.584,0.686}{\textit{#1}}}%
\newcommand{\hlopt}[1]{\textcolor[rgb]{0,0,0}{#1}}%
\newcommand{\hlstd}[1]{\textcolor[rgb]{0.345,0.345,0.345}{#1}}%
\newcommand{\hlkwa}[1]{\textcolor[rgb]{0.161,0.373,0.58}{\textbf{#1}}}%
\newcommand{\hlkwb}[1]{\textcolor[rgb]{0.69,0.353,0.396}{#1}}%
\newcommand{\hlkwc}[1]{\textcolor[rgb]{0.333,0.667,0.333}{#1}}%
\newcommand{\hlkwd}[1]{\textcolor[rgb]{0.737,0.353,0.396}{\textbf{#1}}}%

\usepackage{framed}
\makeatletter
\newenvironment{kframe}{%
 \def\at@end@of@kframe{}%
 \ifinner\ifhmode%
  \def\at@end@of@kframe{\end{minipage}}%
  \begin{minipage}{\columnwidth}%
 \fi\fi%
 \def\FrameCommand##1{\hskip\@totalleftmargin \hskip-\fboxsep
 \colorbox{shadecolor}{##1}\hskip-\fboxsep
     % There is no \\@totalrightmargin, so:
     \hskip-\linewidth \hskip-\@totalleftmargin \hskip\columnwidth}%
 \MakeFramed {\advance\hsize-\width
   \@totalleftmargin\z@ \linewidth\hsize
   \@setminipage}}%
 {\par\unskip\endMakeFramed%
 \at@end@of@kframe}
\makeatother

\definecolor{shadecolor}{rgb}{.97, .97, .97}
\definecolor{messagecolor}{rgb}{0, 0, 0}
\definecolor{warningcolor}{rgb}{1, 0, 1}
\definecolor{errorcolor}{rgb}{1, 0, 0}
\newenvironment{knitrout}{}{} % an empty environment to be redefined in TeX

\usepackage{alltt}
\unitcode{MATH1005}
        \unitname{Statistics}
        \semester{Summer/Winter/Semester2}
        %\sheetnumber1
\IfFileExists{upquote.sty}{\usepackage{upquote}}{}
\begin{document}
\lettersfirst

\begin{tutorial}

You can bring your laptop to use in Labs, providing it has R and RStudio installed (see {\bf Intro to R}). However for the assessable Quizzes, you will need to use the Lab computers.

\vspace{.5cm}
To login follow these steps:

\begin{enumerate}
\item {\bf Login to Zeno}  \\
Type in your Unikey information.  \\
Note: If the computer is not set up for Zeno, change it to Zeno on the desktop. \\

\item {\bf Open RStudio} \\
Bring up the fluxbox menu by right clicking the mouse while the cursor is on the grey background, and then click on \texttt{RStudio}. \\

\item {\bf Open up Firefox and open  maths.usyd.edu.au/MATH1005/} \\
Now you can can access the tutorial sheets and data files. \\
Note: Firefox may come up automatically when you login. Otherwise, you can find it in the fluxbox. \\

\item {\bf Rearrange your desk top} \\
Resize your windows so that you can work in both R and Firefox concurrently. \\
Note: Close the X Terminal window, which opens up automatically on login. \\

\item {\bf Log Off} \\
When it comes time to log off, use the fluxbox.

\end{enumerate}





\end{tutorial}
\end{document}

